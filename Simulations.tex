\iffalse
NOTES TO SELF:
physics motivation
data sets before event selection
PID
exclusive cuts
results to show
1 slide of what needs to go from here to cross section
\fi

\section{Generator and Simulations}
    Simulations are necessary in order to extract correction factors. Presently, only an acceptance correction using a non-radiative generator has been calculated; other correction factors are forthcoming but will not be including in this note. 

    GEMC was used to process generated events through the CLAS12 fall 2018 RG-A configuration. Specifically, a generator based off the GK model and CLAS6 data - aao\_norad\footnote{https://github.com/drewkenjo/aao\_norad}. 
    

\section{Comparison to Data: Missing Mass Distributions}

The standard aao simulations result in missing mass distributions that are too optimistic compared to experimental data. Observe the discrepancies between simulated and experimental distributions in figure \ref{fig:bad}. 


\begin{figure}[hbt]
	\centering
	\includegraphics[page=125,width=0.3\linewidth]{simcomp/nosmear/outbending_rad_All_All_All_no_smearingME_epgg_exp_vs_sim.png}
	\includegraphics[page=123,width=0.3\linewidth]{simcomp/nosmear/outbending_rad_All_All_All_no_smearingMM2_egg_exp_vs_sim.png}
	\includegraphics[page=128,width=0.3\linewidth]{simcomp/nosmear/outbending_rad_All_All_All_no_smearingMM2_epgg_exp_vs_sim.png}
	\includegraphics[page=130,width=0.3\linewidth]{simcomp/nosmear/outbending_rad_All_All_All_no_smearingMM2_ep_exp_vs_sim.png}
	\includegraphics[page=133,width=0.3\linewidth]{simcomp/nosmear/outbending_rad_All_All_All_no_smearingMpi0_exp_vs_sim.png}
	\includegraphics[page=135,width=0.3\linewidth]{simcomp/nosmear/outbending_rad_All_All_All_no_smearingMPt_exp_vs_sim.png}
	
	\caption{Comparison of experiment (blue) and simulation (red) missing mass, energy, momentum, and invariant gamma-gamma mass distributions, before any smearing factors were added to the simulation data.}
	\label{fig:bad}
\end{figure}



To improve the matching between simulation and experiment, gaussian smearing factors were added after reconstruction to the simulated dataset. These factors were tuned by Sangbaek Lee to have optimal matching across all missing mass spectra combinations (figure \ref{fig:good}. Once these factors were determined, the simulations were used to extract an acceptance correction.


\begin{figure}[hbt]
	\centering
	\includegraphics[page=125,width=0.3\linewidth]{simcomp/yessmear/outbending_rad_All_All_All_for_aps_2022_plots_sangcutsME_epgg_exp_vs_sim.png}
	\includegraphics[page=123,width=0.3\linewidth]{simcomp/yessmear/outbending_rad_All_All_All_for_aps_2022_plots_sangcutsMM2_egg_exp_vs_sim.png}
	\includegraphics[page=128,width=0.3\linewidth]{simcomp/yessmear/outbending_rad_All_All_All_for_aps_2022_plots_sangcutsMM2_epgg_exp_vs_sim.png}
	\includegraphics[page=130,width=0.3\linewidth]{simcomp/yessmear/outbending_rad_All_All_All_for_aps_2022_plots_sangcutsMM2_ep_exp_vs_sim.png}
	\includegraphics[page=133,width=0.3\linewidth]{simcomp/yessmear/outbending_rad_All_All_All_for_aps_2022_plots_sangcutsMpi0_exp_vs_sim.png}
	\includegraphics[page=135,width=0.3\linewidth]{simcomp/yessmear/outbending_rad_All_All_All_for_aps_2022_plots_sangcutsMPt_exp_vs_sim.png}
	
	\caption{Comparison of experiment (blue) and simulation (red) missing mass, energy, momentum, and invariant gamma-gamma mass distributions, with smearing factors added to the simulation data proton and photon momenta.}
	\label{fig:good}
\end{figure}


\section{Acceptance Correction}

The acceptance correction, in each kinematic bin, was simply calculated as $\frac{N_{rec}}{N_{gen}}$, where $N_{gen}$ is the number of events produced by the generator in a specific bin, and $N_{rec}$ is the number of events in that bin that were reconstructed and identified as a $DV\pi^0P$ event by exclusivity cuts. 

Low acceptance bins (less than 5\%) were excluded from further study, following from the CLAS6 analysis procedure. 