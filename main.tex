\documentclass[11pt]{report}
\usepackage{outline}
\usepackage[utf8]{inputenc}
\usepackage{graphicx}
\usepackage[top=1in,bottom=1in,left=1in,right=1in]{geometry}
\usepackage{wrapfig}
\usepackage[most]{tcolorbox}
\usepackage[colorlinks = true,
            linkcolor = blue,
            urlcolor  = blue,
            citecolor = blue,
            anchorcolor = blue]{hyperref}
%\usepackage{fancyhdr}
\usepackage{subcaption}
\usepackage{apacite}
\bibliographystyle{apacite}
\usepackage[toc,page]{appendix}
\usepackage{grffile}
\newcommand{\Lumi}{ \mathcal{L}}


\title{Analysis Note on the Deeply Virtual Neutral Pion Production Cross Section Measurement}


\author{Robert Johnston$^1$ \and Sangbaek Lee$^1$
\and Igor Korover$^1$
\and Xiaqing Li$^1$
\and Andrey Kim$^2$
}
\date{%
    $^1$Massachusetts Institute of Technology\\%
    $^2$University of Connecticut\\[2ex]%
    \today
}


\begin{document}

\maketitle
\tableofcontents
\cleardoublepage
% \phantomsection
\addcontentsline{toc}{chapter}{\listfigurename}
\listoffigures

\chapter{Collaborator's Note: Differences Between This and Prior Analysis Note}
For ease of reading by collaborators, here are a listing of differences between this analysis note (for Fall 2022 DNP meeting) and the most recently collaboration approved analysis note (for April 2022 APS meeting):
\begin{itemize}
    \item \textbf{Model Comparison} The large reported change from analysis notes is the inclusion of the Goloskov Kroll Model to compare the preliminary CLAS12 cross section to a theoretical model. This is covered in section \ref{chap:GKmodel}.
    \item \textbf{Exclusivity Cuts and Results} There are small improvements to this text for a more complete description of work accomplished.
\end{itemize}

\chapter{PID}
\input{PID}

\chapter{Exclusive Selection}  
\label{chap:exclusive}
\input{Exclusive}

\chapter{Simulations}
\label{chap:acc}
\iffalse
NOTES TO SELF:
physics motivation
data sets before event selection
PID
exclusive cuts
results to show
1 slide of what needs to go from here to cross section
\fi

\section{Generator and Simulations}
    Simulations are necessary in order to extract correction factors. Presently, only an acceptance correction using a non-radiative generator has been calculated; other correction factors are forthcoming but will not be including in this note. 

    GEMC was used to process generated events through the CLAS12 fall 2018 RG-A configuration. Specifically, a generator based off the GK model and CLAS6 data - aao\_norad\footnote{https://github.com/drewkenjo/aao\_norad}. 
    

\section{Comparison to Data: Missing Mass Distributions}

The standard aao simulations result in missing mass distributions that are too optimistic compared to experimental data. Observe the discrepancies between simulated and experimental distributions in figure \ref{fig:bad}. 


\begin{figure}[hbt]
	\centering
	\includegraphics[page=125,width=0.3\linewidth]{simcomp/nosmear/outbending_rad_All_All_All_no_smearingME_epgg_exp_vs_sim.png}
	\includegraphics[page=123,width=0.3\linewidth]{simcomp/nosmear/outbending_rad_All_All_All_no_smearingMM2_egg_exp_vs_sim.png}
	\includegraphics[page=128,width=0.3\linewidth]{simcomp/nosmear/outbending_rad_All_All_All_no_smearingMM2_epgg_exp_vs_sim.png}
	\includegraphics[page=130,width=0.3\linewidth]{simcomp/nosmear/outbending_rad_All_All_All_no_smearingMM2_ep_exp_vs_sim.png}
	\includegraphics[page=133,width=0.3\linewidth]{simcomp/nosmear/outbending_rad_All_All_All_no_smearingMpi0_exp_vs_sim.png}
	\includegraphics[page=135,width=0.3\linewidth]{simcomp/nosmear/outbending_rad_All_All_All_no_smearingMPt_exp_vs_sim.png}
	
	\caption{Comparison of experiment (blue) and simulation (red) missing mass, energy, momentum, and invariant gamma-gamma mass distributions, before any smearing factors were added to the simulation data.}
	\label{fig:bad}
\end{figure}



To improve the matching between simulation and experiment, gaussian smearing factors were added after reconstruction to the simulated dataset. These factors were tuned by Sangbaek Lee to have optimal matching across all missing mass spectra combinations (figure \ref{fig:good}. Once these factors were determined, the simulations were used to extract an acceptance correction.


\begin{figure}[hbt]
	\centering
	\includegraphics[page=125,width=0.3\linewidth]{simcomp/yessmear/outbending_rad_All_All_All_for_aps_2022_plots_sangcutsME_epgg_exp_vs_sim.png}
	\includegraphics[page=123,width=0.3\linewidth]{simcomp/yessmear/outbending_rad_All_All_All_for_aps_2022_plots_sangcutsMM2_egg_exp_vs_sim.png}
	\includegraphics[page=128,width=0.3\linewidth]{simcomp/yessmear/outbending_rad_All_All_All_for_aps_2022_plots_sangcutsMM2_epgg_exp_vs_sim.png}
	\includegraphics[page=130,width=0.3\linewidth]{simcomp/yessmear/outbending_rad_All_All_All_for_aps_2022_plots_sangcutsMM2_ep_exp_vs_sim.png}
	\includegraphics[page=133,width=0.3\linewidth]{simcomp/yessmear/outbending_rad_All_All_All_for_aps_2022_plots_sangcutsMpi0_exp_vs_sim.png}
	\includegraphics[page=135,width=0.3\linewidth]{simcomp/yessmear/outbending_rad_All_All_All_for_aps_2022_plots_sangcutsMPt_exp_vs_sim.png}
	
	\caption{Comparison of experiment (blue) and simulation (red) missing mass, energy, momentum, and invariant gamma-gamma mass distributions, with smearing factors added to the simulation data proton and photon momenta.}
	\label{fig:good}
\end{figure}


\section{Acceptance Correction}

The acceptance correction, in each kinematic bin, was simply calculated as $\frac{N_{rec}}{N_{gen}}$, where $N_{gen}$ is the number of events produced by the generator in a specific bin, and $N_{rec}$ is the number of events in that bin that were reconstructed and identified as a $DV\pi^0P$ event by exclusivity cuts. 

Low acceptance bins (less than 5\%) were excluded from further study, following from the CLAS6 analysis procedure. 

\chapter{Luminosity and Virtual Photon Flux Factor}
\label{chap:lumi}
\input{Luminosity}

\chapter{Results}
\input{Results}

\chapter{Comparison with Model} 
\label{chap:GKmodel}
\input{model}


\chapter{Acknowledgements}
Acknowledgement should be given to the MIT Milner Hadronic Physics group: Richard Milner, Douglas Hasell, Igor Korover, Xiaqing Li, Patrick Moran, Sangbaek Lee, and Robert Johnston. This work was also facilitated by the use of OSG and MIT Tier 2 computing, so we would like to thank Maurizio Ungaro, Christoph Paus, Ernie Ihloff, Jim Kelsey, and others for their technical support. 

\include{biblio}

\chapter{Appendix}

\iffalse

From paper on understanding pi+ production, we have:

 \begin{equation}\label{xsec}
     \frac{d^4\sigma_{\gamma^*p \rightarrow p'\pi^0}}{dQ^2W^2dtd\phi_{\pi}} =
     \frac{\alpha (W^2-m^2)}{16\pi^2 E^2_L m^2 Q^2 (1-\epsilon)}
     ((\frac{d\sigma_T}{dt}+\epsilon\frac{d\sigma_L}{dt})+
     \epsilon cos(2\phi) \frac{d\sigma_{TT}}{dt} + \sqrt{2\epsilon(1+\epsilon)}cos(\phi)\frac{d\sigma_{LT}}{dt})
\end{equation}

Comparing the two, we have a difference in the prefactor of:



0.3894 * 1E6 * $\frac{1}{16\pi(W^2-m_p^2)\sqrt{W^4 + (Q^2)^2+m_p^4+2W^2Q^2-2W^2m_p^2+2Q^2m_p^2}}$
\fi

\section{Cross check between Andrey Kim and Bobby Johnston}

As an additional cross check, Bobby calculated a $DV\pi^0P$ beam spin asymmetry and compared to Andrey Kim's results. This check will not comment on any acceptance, luminosity, or virtual photon flux factor calculations, but does validate exclusive event selection criteria. By examining figure \ref{fig:bsa} we can see that agreement is reasonable, especially considering Bobby's calculation does not have sideband subtraction included.

\begin{figure}[hbt]
	\centering
	\includegraphics[width=0.75\linewidth]{BSA.png}
	
	
	\caption{Overlay comparison of Andrey Kim's results (black datapoints, red fit line) and Bobby's results (red datapoints, orange fit line).}
	\label{fig:good}
\end{figure}

%\section{Proton PID lookback}
%We extended the analysis to positive tracks.
%Fig.~\ref{fig:finaldatapid} shows the exclusive distributions for positive tracks that were not identified as protons.
%It seems that they are valid exclusive $\pi^0$ electroproduction events even though the positive track was not identified as proton.
%Most of the misidentified tracks happen to be reconstructed in Central Detector.
%Based of Fig.~\ref{fig:finaldatapid} we can make a conclusion that exclusive cuts that we are able to apply because of detection of all final state particles allow us to clean up our sample even without proton PID cuts on positive tracks.
%There seems to be more background for the events with misidentified proton so the tracking improvements (central in particular) would help to clean up the sample in the future.

%\begin{figure}[h]
%    \centering
%    \includegraphics[page=3,width=0.4\linewidth]{figures/eppi0_misc.pdf}
%    \includegraphics[page=4,width=0.4\linewidth]{figures/eppi0_misc.pdf}
%\caption{On the left: is the distribution of $MM^2_{epX}$ for non-proton positive tracks that would pass exclusivity cuts.
%On the right: polar angle for protons in FD (blue), protons in CD (green), non-proton positives in CD (red).}
%    \label{fig:finaldatapid}s
%\end{figure}



%\section{BACKUP}
%\noindent

%\begin{tcolorbox}
%\includegraphics[page=1,width=0.48\linewidth]{figures/bsa_eppi0_ge_pro_fd.pdf}
%\includegraphics[page=1,width=0.48\linewidth]{figures/bsa_eppi0_ge_pro_cd.pdf}
%\end{tcolorbox}

%\begin{tcolorbox}
%\includegraphics[page=2,width=0.48\linewidth]{figures/bsa_eppi0_ge_pro_fd.pdf}
%\includegraphics[page=2,width=0.48\linewidth]{figures/bsa_eppi0_ge_pro_cd.pdf}
%\end{tcolorbox}

%\begin{tcolorbox}
%\includegraphics[page=3,width=0.48\linewidth]{figures/bsa_eppi0_ge_pro_fd.pdf}
%\includegraphics[page=3,width=0.48\linewidth]{figures/bsa_eppi0_ge_pro_cd.pdf}
%\end{tcolorbox}

%\begin{tcolorbox}
%\includegraphics[page=4,width=0.48\linewidth]{figures/bsa_eppi0_ge_pro_fd.pdf}
%\includegraphics[page=4,width=0.48\linewidth]{figures/bsa_eppi0_ge_pro_cd.pdf}
%\end{tcolorbox}

%\begin{tcolorbox}
%\includegraphics[page=5,width=0.48\linewidth]{figures/bsa_eppi0_ge_pro_fd.pdf}
%\includegraphics[page=5,width=0.48\linewidth]{figures/bsa_eppi0_ge_pro_cd.pdf}
%\end{tcolorbox}

%\begin{tcolorbox}
%\includegraphics[page=6,width=0.48\linewidth]{figures/bsa_eppi0_ge_pro_fd.pdf}
%\includegraphics[page=6,width=0.48\linewidth]{figures/bsa_eppi0_ge_pro_cd.pdf}
%\end{tcolorbox}

%\begin{tcolorbox}
%\includegraphics[page=7,width=0.48\linewidth]{figures/bsa_eppi0_ge_pro_fd.pdf}
%\includegraphics[page=7,width=0.48\linewidth]{figures/bsa_eppi0_ge_pro_cd.pdf}
%\end{tcolorbox}




\end{document}
